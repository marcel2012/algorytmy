\documentclass[a4paper,12pt,twoside]{book}
\usepackage[letterpaper, margin=1in]{geometry}
\usepackage[utf8]{inputenc}
\usepackage[T1]{fontenc}
\usepackage[MeX]{polski}
\usepackage[table]{xcolor}
\usepackage{color}
\usepackage{latexsym}
\author{Kacper Nowicki, Marcel Korpal}
\title{Generacje Komputerów}
\frenchspacing
\usepackage{fancyhdr}
\usepackage{graphicx} 
 \usepackage{caption}
\pagestyle{fancy}
\fancyhf{}
\makeatletter
\let\ps@plain\ps@fancy
\makeatother
\fancyhead[LE]{\leftmark}
\fancyhead[RO]{Kacper Nowicki, Marcel Korpal}
\fancyfoot[C]{\thepage}
\begin{document}
\fontfamily{qhv}\selectfont\mdseries
\maketitle
\tableofcontents
\chapter{Zerowa Generacja}
Komputery zerowej generacji to maszyny konstruowane przed pojawieniem się uniwersalnych, elektronicznych maszyn cyfrowych, o możliwościach dzisiejszych prostych i średnich kalkulatorów programowanych.

Podstawową ich cechą jest brak aktywnych elementów elektronicznych (lamp i tranzystorów). Budowane były na elementach mechanicznych (np. Z1) lub elektromagnetycznych (np. przekaźnikowy Z3\footnote{Przekaźnikowy Z3 to pierwszy działający, w pełni automatyczny komputer o zmiennym programie zbudowany przez niemieckiego inżyniera Konrada Zuse w 1941 roku na bazie jego wcześniejszej, mechanicznej konstrukcji, Z1. Maszyna była wykorzystywana w czasie wojny do obliczeń niezbędnych przy projektowaniu.}). Do budowy próbowano także wykorzystać gotowe arytmometry elektromechaniczne np. maszynę do fakturowania w PARK. 
\newpage
\begin{figure}
\centering
\includegraphics[width=10cm]1
\caption{Replika Z1}
\label{fig:obrazek 1}
\end{figure}

\chapter{Pierwsza Generacja}
Komputer pierwszej generacji to komputer zbudowany na lampach elektronowych.
\newpage
\begin{figure}
\centering
\includegraphics[width=10cm]2
\caption{XYZ - pierwszy polski komputer}
\label{fig:obrazek 2}
\end{figure}
\chapter{Druga Generacja}
Komputer drugiej generacji jest to komputer, w którym do budowy elementów logicznych (bramek) wykorzystano elementy półprzewodnikowe – były nimi wynalezione w połowie XX w. tranzystory.
\newpage
\begin{figure}
\centering
\includegraphics[width=10cm]3
\caption{Komputer ZAM-41}
\label{fig:obrazek 3}
\end{figure}

\chapter{Trzecia Generacja}
Komputer trzeciej generacji – komputer zbudowany na układach scalonych małej i średniej skali integracji, które zawierały tylko kilka do kilkunastu struktur półprzewodnikowych na jednej płytce. Trzecią generację rozpoczęła seria 360 komputerów przedsiębiorstwa IBM. Komputery te były budowane w latach 1965-1980. Polski komputer trzeciej generacji to Odra 1305.
\newpage

\begin{figure}
\centering
\includegraphics[width=10cm]4
\caption{Odra 1305}
\label{fig:obrazek 4}
\end{figure}

\chapter{Czwarta Generacja}
Komputer czwartej generacji to komputer zbudowany na układach scalonych wielkiej skali integracji\cite{p1} np. komputer osobisty (PC).
\newpage

\begin{figure}
\centering
\includegraphics[width=10cm]5
\caption{Podstawowe komponenty komputera osobistego:\\
1) Monitor\\
2) płyta główna\\
3) procesor (CPU)\\
4) pamięć operacyjna (RAM)\\
5) karta rozszerzenia\\
6) zasilacz\\
7) napęd optyczny (CD, DVD itp.)\\
8) dysk twardy (HDD)\\
9) mysz\\
10) klawiatura\\}
\label{fig:obrazek 5}
\end{figure}

\chapter{Piąta Generacja}
Komputery piątej generacji to projekty o niekonwencjonalnych rozwiązaniach, np. komputer kwantowy.
\newpage

\begin{figure}
\centering
\includegraphics[width=10cm]6
\caption{Układ skonstruowany przez D-Wave Systems, zrealizowanych za pomocą nadprzewodników}
\label{fig:obrazek 6}
\end{figure}







\chapter{Dane}


\centering
\begin{table}[ht]
\caption{Rok pojawienia się generacji}
\centering
\begin{tabular}{|c|c|}
\hline
Zerowa Generacja & 1936 \\
\hline
Pierwsza Generacja & 1939  \\
\hline
Druga Generacja & 1961  \\
\hline
Trzecia Generacja & 1965  \\
\hline
Czwarta Generacja & 1970  \\
\hline
Piąta Generacja & BRAK  \\
\hline
\end{tabular}
\end{table}







\begin{thebibliography}{99}
\bibitem{p1} Tytuł:
\emph{Technika. Spojrzenie na dzieje cywilizacji.. Warszawa: PWN, 2003, s. 80.}
\end{thebibliography}



\listoffigures
\addcontentsline{toc}{section}{Spis rysunków}

\listoftables

\end{document}
